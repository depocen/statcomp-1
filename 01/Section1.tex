\documentclass[8pt,ignorenonframetext,]{beamer}
\setbeamertemplate{caption}[numbered]
\setbeamertemplate{caption label separator}{: }
\setbeamercolor{caption name}{fg=normal text.fg}
\beamertemplatenavigationsymbolsempty
\usepackage{lmodern}
\usepackage{amssymb,amsmath}
\usepackage{ifxetex,ifluatex}
\usepackage{fixltx2e} % provides \textsubscript
\ifnum 0\ifxetex 1\fi\ifluatex 1\fi=0 % if pdftex
  \usepackage[T1]{fontenc}
  \usepackage[utf8]{inputenc}
\else % if luatex or xelatex
  \ifxetex
    \usepackage{mathspec}
  \else
    \usepackage{fontspec}
  \fi
  \defaultfontfeatures{Ligatures=TeX,Scale=MatchLowercase}
\fi
\usetheme[]{SDATbeamer}
% use upquote if available, for straight quotes in verbatim environments
\IfFileExists{upquote.sty}{\usepackage{upquote}}{}
% use microtype if available
\IfFileExists{microtype.sty}{%
\usepackage{microtype}
\UseMicrotypeSet[protrusion]{basicmath} % disable protrusion for tt fonts
}{}
\newif\ifbibliography
\hypersetup{
            pdftitle={Lecture 1: Introduction to the Course Basics of Data},
            pdfauthor={S. Morteza Najibi},
            pdfborder={0 0 0},
            breaklinks=true}
\urlstyle{same}  % don't use monospace font for urls
\usepackage{color}
\usepackage{fancyvrb}
\newcommand{\VerbBar}{|}
\newcommand{\VERB}{\Verb[commandchars=\\\{\}]}
\DefineVerbatimEnvironment{Highlighting}{Verbatim}{commandchars=\\\{\}}
% Add ',fontsize=\small' for more characters per line
\usepackage{framed}
\definecolor{shadecolor}{RGB}{248,248,248}
\newenvironment{Shaded}{\begin{snugshade}}{\end{snugshade}}
\newcommand{\KeywordTok}[1]{\textcolor[rgb]{0.13,0.29,0.53}{\textbf{#1}}}
\newcommand{\DataTypeTok}[1]{\textcolor[rgb]{0.13,0.29,0.53}{#1}}
\newcommand{\DecValTok}[1]{\textcolor[rgb]{0.00,0.00,0.81}{#1}}
\newcommand{\BaseNTok}[1]{\textcolor[rgb]{0.00,0.00,0.81}{#1}}
\newcommand{\FloatTok}[1]{\textcolor[rgb]{0.00,0.00,0.81}{#1}}
\newcommand{\ConstantTok}[1]{\textcolor[rgb]{0.00,0.00,0.00}{#1}}
\newcommand{\CharTok}[1]{\textcolor[rgb]{0.31,0.60,0.02}{#1}}
\newcommand{\SpecialCharTok}[1]{\textcolor[rgb]{0.00,0.00,0.00}{#1}}
\newcommand{\StringTok}[1]{\textcolor[rgb]{0.31,0.60,0.02}{#1}}
\newcommand{\VerbatimStringTok}[1]{\textcolor[rgb]{0.31,0.60,0.02}{#1}}
\newcommand{\SpecialStringTok}[1]{\textcolor[rgb]{0.31,0.60,0.02}{#1}}
\newcommand{\ImportTok}[1]{#1}
\newcommand{\CommentTok}[1]{\textcolor[rgb]{0.56,0.35,0.01}{\textit{#1}}}
\newcommand{\DocumentationTok}[1]{\textcolor[rgb]{0.56,0.35,0.01}{\textbf{\textit{#1}}}}
\newcommand{\AnnotationTok}[1]{\textcolor[rgb]{0.56,0.35,0.01}{\textbf{\textit{#1}}}}
\newcommand{\CommentVarTok}[1]{\textcolor[rgb]{0.56,0.35,0.01}{\textbf{\textit{#1}}}}
\newcommand{\OtherTok}[1]{\textcolor[rgb]{0.56,0.35,0.01}{#1}}
\newcommand{\FunctionTok}[1]{\textcolor[rgb]{0.00,0.00,0.00}{#1}}
\newcommand{\VariableTok}[1]{\textcolor[rgb]{0.00,0.00,0.00}{#1}}
\newcommand{\ControlFlowTok}[1]{\textcolor[rgb]{0.13,0.29,0.53}{\textbf{#1}}}
\newcommand{\OperatorTok}[1]{\textcolor[rgb]{0.81,0.36,0.00}{\textbf{#1}}}
\newcommand{\BuiltInTok}[1]{#1}
\newcommand{\ExtensionTok}[1]{#1}
\newcommand{\PreprocessorTok}[1]{\textcolor[rgb]{0.56,0.35,0.01}{\textit{#1}}}
\newcommand{\AttributeTok}[1]{\textcolor[rgb]{0.77,0.63,0.00}{#1}}
\newcommand{\RegionMarkerTok}[1]{#1}
\newcommand{\InformationTok}[1]{\textcolor[rgb]{0.56,0.35,0.01}{\textbf{\textit{#1}}}}
\newcommand{\WarningTok}[1]{\textcolor[rgb]{0.56,0.35,0.01}{\textbf{\textit{#1}}}}
\newcommand{\AlertTok}[1]{\textcolor[rgb]{0.94,0.16,0.16}{#1}}
\newcommand{\ErrorTok}[1]{\textcolor[rgb]{0.64,0.00,0.00}{\textbf{#1}}}
\newcommand{\NormalTok}[1]{#1}

% Prevent slide breaks in the middle of a paragraph:
\widowpenalties 1 10000
\raggedbottom

\AtBeginPart{
  \let\insertpartnumber\relax
  \let\partname\relax
  \frame{\partpage}
}
\AtBeginSection{
  \ifbibliography
  \else
    \let\insertsectionnumber\relax
    \let\sectionname\relax
    \frame{\sectionpage}
  \fi
}
\AtBeginSubsection{
  \let\insertsubsectionnumber\relax
  \let\subsectionname\relax
  \frame{\subsectionpage}
}

\setlength{\parindent}{0pt}
\setlength{\parskip}{6pt plus 2pt minus 1pt}
\setlength{\emergencystretch}{3em}  % prevent overfull lines
\providecommand{\tightlist}{%
  \setlength{\itemsep}{0pt}\setlength{\parskip}{0pt}}
\setcounter{secnumdepth}{0}

\title{Lecture 1: Introduction to the Course Basics of Data}
\author{S. Morteza Najibi}
\institute{Shiraz University}
\date{February 16, 2019}

\begin{document}
\frame{\titlepage}

\section{Introduction}\label{introduction}

\begin{frame}{Agenda}

\begin{itemize}
\tightlist
\item
  Course overview and mechanics
\item
  Built-in data types
\item
  Built-in functions and operators
\item
  First data structures: Vectors and arrays
\end{itemize}

\end{frame}

\begin{frame}{Why good statisticians learn to program}

\begin{itemize}
\item
  \emph{Independence}: Otherwise, you rely on someone else having given
  you exactly the right tool
\item
  \emph{Honesty}: Otherwise, you end up distorting your problem to match
  the tools you have
\item
  \emph{Clarity}: Making your method something a machine can do
  disciplines your thinking and makes it public; that's science
\end{itemize}

\end{frame}

\begin{frame}{How this class will work}

\begin{itemize}
\tightlist
\item
  No programming knowledge presumed
\item
  Some stats. knowledge presumed
\item
  General programming mixed with data-manipulation and statistical
  inference
\item
  Class will be \emph{very} cumulative
\item
  Keep up with the readings and assignments!
\end{itemize}

\end{frame}

\begin{frame}{assignment}

Assignments, class notes, grading policies, useful links on R:

\begin{itemize}
\tightlist
\item
  \url{http://cw.sdat.ir}
\end{itemize}

\end{frame}

\section{R Basics}\label{r-basics}

\begin{frame}[fragile]{The class in a nutshell: Functional programming}

2 sorts of things (\textbf{objects}): \textbf{data} and
\textbf{functions}

\begin{itemize}
\item
  \textbf{Data}: things like 7, ``seven'', \(7.000\), the matrix
  \(\left[ \begin{array}{ccc} 7 & 7 & 7 \\ 7 & 7 & 7\end{array}\right]\)
\item
  \textbf{Functions}: things like \(\log{}\), \(+\) (two arguments),
  \(<\) (two), \(\mod{}\) (two), \texttt{mean} (one)
\end{itemize}

\begin{quote}
A function is a machine which turns input objects (\textbf{arguments})
into an output object (\textbf{return value}), possibly with
\textbf{side effects}, according to a definite rule
\end{quote}

\end{frame}

\begin{frame}{Continue}

Programming is writing functions to transform inputs into outputs

Good programming ensures the transformation is done easily and correctly

Machines are made out of machines; functions are made out of functions,
like \(f(a,b) = a^2 + b^2\)

\begin{quote}
The route to good programming is to take the big transformation and
break it down into smaller ones, and then break those down, until you
come to tasks which the built-in functions can do
\end{quote}

\end{frame}

\begin{frame}[fragile]{Before functions, data}

Different kinds of data object

All data is represented in binary format, by \textbf{bits} (TRUE/FALSE,
YES/NO, 1/0)

\begin{itemize}
\tightlist
\item
  \textbf{Booleans} Direct binary values: \texttt{TRUE} or
  \texttt{FALSE} in R
\item
  \textbf{Integers}: whole numbers (positive, negative or zero),
  represented by a fixed-length block of bits
\item
  \textbf{Characters} fixed-length blocks of bits, with special coding;
  \textbf{strings} = sequences of characters
\item
  \textbf{Floating point numbers}: a fraction (with a finite number of
  bits) times an exponent, like \(1.87 \times {10}^{6}\), but in binary
  form
\item
  \textbf{Missing or ill-defined values}: \texttt{NA}, \texttt{NaN},
  etc.
\end{itemize}

\end{frame}

\begin{frame}[fragile]{Operators}

\begin{itemize}
\tightlist
\item
  \textbf{Unary} \texttt{-} for arithmetic negation, \texttt{!} for
  Boolean
\item
  \textbf{Binary} usual arithmetic operators, plus ones for modulo and
  integer division; take two numbers and give a number
\end{itemize}

\end{frame}

\begin{frame}[fragile]{some basics}

\begin{Shaded}
\begin{Highlighting}[]
\DecValTok{7}\OperatorTok{+}\DecValTok{5}
\end{Highlighting}
\end{Shaded}

\begin{verbatim}
## [1] 12
\end{verbatim}

\begin{Shaded}
\begin{Highlighting}[]
\DecValTok{7}\OperatorTok{-}\DecValTok{5}
\end{Highlighting}
\end{Shaded}

\begin{verbatim}
## [1] 2
\end{verbatim}

\begin{Shaded}
\begin{Highlighting}[]
\DecValTok{7}\OperatorTok{*}\DecValTok{5}
\end{Highlighting}
\end{Shaded}

\begin{verbatim}
## [1] 35
\end{verbatim}

\begin{Shaded}
\begin{Highlighting}[]
\DecValTok{7}\OperatorTok{^}\DecValTok{5}
\end{Highlighting}
\end{Shaded}

\begin{verbatim}
## [1] 16807
\end{verbatim}

\end{frame}

\begin{frame}[fragile]{Continue}

\begin{Shaded}
\begin{Highlighting}[]
\DecValTok{7}\OperatorTok{/}\DecValTok{5}
\end{Highlighting}
\end{Shaded}

\begin{verbatim}
## [1] 1.4
\end{verbatim}

\begin{Shaded}
\begin{Highlighting}[]
\DecValTok{7} \OperatorTok\StringTok{ }\DecValTok{5}
\end{Highlighting}
\end{Shaded}

\begin{verbatim}
## [1] 2
\end{verbatim}

\begin{Shaded}
\begin{Highlighting}[]
\DecValTok{7} \OperatorTok\StringTok{ }\DecValTok{5}
\end{Highlighting}
\end{Shaded}

\begin{verbatim}
## [1] 1
\end{verbatim}

\end{frame}

\begin{frame}{The R Console}

Basic interaction with R is by typing in the \textbf{console}, a.k.a.
\textbf{terminal} or \textbf{command-line}

You type in commands, R gives back answers (or errors)

Menus and other graphical interfaces are extras built on top of the
console

\end{frame}

\begin{frame}[fragile]{Operators cont'd.}

\textbf{Comparisons} are also binary operators; they take two objects,
like numbers, and give a Boolean

\begin{Shaded}
\begin{Highlighting}[]
\DecValTok{7} \OperatorTok{>}\StringTok{ }\DecValTok{5}
\end{Highlighting}
\end{Shaded}

\begin{verbatim}
## [1] TRUE
\end{verbatim}

\begin{Shaded}
\begin{Highlighting}[]
\DecValTok{7} \OperatorTok{<}\StringTok{ }\DecValTok{5}
\end{Highlighting}
\end{Shaded}

\begin{verbatim}
## [1] FALSE
\end{verbatim}

\begin{Shaded}
\begin{Highlighting}[]
\DecValTok{7} \OperatorTok{>=}\StringTok{ }\DecValTok{7}
\end{Highlighting}
\end{Shaded}

\begin{verbatim}
## [1] TRUE
\end{verbatim}

\begin{Shaded}
\begin{Highlighting}[]
\DecValTok{7} \OperatorTok{<=}\StringTok{ }\DecValTok{5}
\end{Highlighting}
\end{Shaded}

\begin{verbatim}
## [1] FALSE
\end{verbatim}

\end{frame}

\begin{frame}[fragile]{}

\begin{Shaded}
\begin{Highlighting}[]
\DecValTok{7} \OperatorTok{==}\StringTok{ }\DecValTok{5}
\end{Highlighting}
\end{Shaded}

\begin{verbatim}
## [1] FALSE
\end{verbatim}

\begin{Shaded}
\begin{Highlighting}[]
\DecValTok{7} \OperatorTok{!=}\StringTok{ }\DecValTok{5}
\end{Highlighting}
\end{Shaded}

\begin{verbatim}
## [1] TRUE
\end{verbatim}

\end{frame}

\begin{frame}[fragile]{Boolean operators}

Basically ``and'' and ``or'':

\begin{Shaded}
\begin{Highlighting}[]
\NormalTok{(}\DecValTok{5} \OperatorTok{>}\StringTok{ }\DecValTok{7}\NormalTok{) }\OperatorTok{&}\StringTok{ }\NormalTok{(}\DecValTok{6}\OperatorTok{*}\DecValTok{7} \OperatorTok{==}\StringTok{ }\DecValTok{42}\NormalTok{)}
\end{Highlighting}
\end{Shaded}

\begin{verbatim}
## [1] FALSE
\end{verbatim}

\begin{Shaded}
\begin{Highlighting}[]
\NormalTok{(}\DecValTok{5} \OperatorTok{>}\StringTok{ }\DecValTok{7}\NormalTok{) }\OperatorTok{|}\StringTok{ }\NormalTok{(}\DecValTok{6}\OperatorTok{*}\DecValTok{7} \OperatorTok{==}\StringTok{ }\DecValTok{42}\NormalTok{)}
\end{Highlighting}
\end{Shaded}

\begin{verbatim}
## [1] TRUE
\end{verbatim}

(will see special doubled forms, \texttt{\&\&} and
\texttt{\textbar{}\textbar{}}, later)

\end{frame}

\begin{frame}[fragile]{More types}

\texttt{typeof()} function returns the type

\texttt{is.}\emph{foo}\texttt{()} functions return Booleans for whether
the argument is of type \emph{foo}

\texttt{as.}\emph{foo}\texttt{()} (tries to) ``cast'' its argument to
type \emph{foo} --- to translate it sensibly into a \emph{foo}-type
value

\end{frame}

\begin{frame}[fragile]{}

\begin{Shaded}
\begin{Highlighting}[]
\KeywordTok{typeof}\NormalTok{(}\DecValTok{7}\NormalTok{)}
\end{Highlighting}
\end{Shaded}

\begin{verbatim}
## [1] "double"
\end{verbatim}

\begin{Shaded}
\begin{Highlighting}[]
\KeywordTok{is.numeric}\NormalTok{(}\DecValTok{7}\NormalTok{)}
\end{Highlighting}
\end{Shaded}

\begin{verbatim}
## [1] TRUE
\end{verbatim}

\begin{Shaded}
\begin{Highlighting}[]
\KeywordTok{is.na}\NormalTok{(}\DecValTok{7}\NormalTok{)}
\end{Highlighting}
\end{Shaded}

\begin{verbatim}
## [1] FALSE
\end{verbatim}

\end{frame}

\begin{frame}[fragile]{}

\begin{Shaded}
\begin{Highlighting}[]
\KeywordTok{is.na}\NormalTok{(}\DecValTok{7}\OperatorTok{/}\DecValTok{0}\NormalTok{)}
\end{Highlighting}
\end{Shaded}

\begin{verbatim}
## [1] FALSE
\end{verbatim}

\begin{Shaded}
\begin{Highlighting}[]
\KeywordTok{is.na}\NormalTok{(}\DecValTok{0}\OperatorTok{/}\DecValTok{0}\NormalTok{)}
\end{Highlighting}
\end{Shaded}

\begin{verbatim}
## [1] TRUE
\end{verbatim}

Why is 7/0 not NA, but 0/0 is?

\end{frame}

\begin{frame}[fragile]{}

\begin{Shaded}
\begin{Highlighting}[]
\KeywordTok{is.character}\NormalTok{(}\DecValTok{7}\NormalTok{)}
\end{Highlighting}
\end{Shaded}

\begin{verbatim}
## [1] FALSE
\end{verbatim}

\begin{Shaded}
\begin{Highlighting}[]
\KeywordTok{is.character}\NormalTok{(}\StringTok{"7"}\NormalTok{)}
\end{Highlighting}
\end{Shaded}

\begin{verbatim}
## [1] TRUE
\end{verbatim}

\begin{Shaded}
\begin{Highlighting}[]
\KeywordTok{is.character}\NormalTok{(}\StringTok{"seven"}\NormalTok{)}
\end{Highlighting}
\end{Shaded}

\begin{verbatim}
## [1] TRUE
\end{verbatim}

\begin{Shaded}
\begin{Highlighting}[]
\KeywordTok{is.na}\NormalTok{(}\StringTok{"seven"}\NormalTok{)}
\end{Highlighting}
\end{Shaded}

\begin{verbatim}
## [1] FALSE
\end{verbatim}

\end{frame}

\begin{frame}[fragile]{}

\begin{Shaded}
\begin{Highlighting}[]
\KeywordTok{as.character}\NormalTok{(}\DecValTok{5}\OperatorTok{/}\DecValTok{6}\NormalTok{)}
\end{Highlighting}
\end{Shaded}

\begin{verbatim}
## [1] "0.833333333333333"
\end{verbatim}

\begin{Shaded}
\begin{Highlighting}[]
\KeywordTok{as.numeric}\NormalTok{(}\KeywordTok{as.character}\NormalTok{(}\DecValTok{5}\OperatorTok{/}\DecValTok{6}\NormalTok{))}
\end{Highlighting}
\end{Shaded}

\begin{verbatim}
## [1] 0.8333333
\end{verbatim}

\begin{Shaded}
\begin{Highlighting}[]
\DecValTok{6}\OperatorTok{*}\KeywordTok{as.numeric}\NormalTok{(}\KeywordTok{as.character}\NormalTok{(}\DecValTok{5}\OperatorTok{/}\DecValTok{6}\NormalTok{))}
\end{Highlighting}
\end{Shaded}

\begin{verbatim}
## [1] 5
\end{verbatim}

\begin{Shaded}
\begin{Highlighting}[]
\DecValTok{5}\OperatorTok{/}\DecValTok{6} \OperatorTok{==}\StringTok{ }\KeywordTok{as.numeric}\NormalTok{(}\KeywordTok{as.character}\NormalTok{(}\DecValTok{5}\OperatorTok{/}\DecValTok{6}\NormalTok{))}
\end{Highlighting}
\end{Shaded}

\begin{verbatim}
## [1] FALSE
\end{verbatim}

(why is that last FALSE?)

\end{frame}

\begin{frame}[fragile]{Data can have names}

We can give names to data objects; these give us \textbf{variables}

A few variables are built in:

\begin{Shaded}
\begin{Highlighting}[]
\NormalTok{pi}
\end{Highlighting}
\end{Shaded}

\begin{verbatim}
## [1] 3.141593
\end{verbatim}

Variables can be arguments to functions or operators, just like
constants:

\begin{Shaded}
\begin{Highlighting}[]
\NormalTok{pi}\OperatorTok{*}\DecValTok{10}
\end{Highlighting}
\end{Shaded}

\begin{verbatim}
## [1] 31.41593
\end{verbatim}

\begin{Shaded}
\begin{Highlighting}[]
\KeywordTok{cos}\NormalTok{(pi)}
\end{Highlighting}
\end{Shaded}

\begin{verbatim}
## [1] -1
\end{verbatim}

\end{frame}

\begin{frame}[fragile]{}

Most variables are created with the \textbf{assignment operator},
\texttt{\textless{}-} or \texttt{=}

\begin{Shaded}
\begin{Highlighting}[]
\NormalTok{approx.pi <-}\StringTok{ }\DecValTok{22}\OperatorTok{/}\DecValTok{7}
\NormalTok{approx.pi}
\end{Highlighting}
\end{Shaded}

\begin{verbatim}
## [1] 3.142857
\end{verbatim}

\begin{Shaded}
\begin{Highlighting}[]
\NormalTok{diameter.in.cubits =}\StringTok{ }\DecValTok{10}
\NormalTok{approx.pi}\OperatorTok{*}\NormalTok{diameter.in.cubits}
\end{Highlighting}
\end{Shaded}

\begin{verbatim}
## [1] 31.42857
\end{verbatim}

\end{frame}

\begin{frame}[fragile]{The assignment operator also changes values:}

\begin{Shaded}
\begin{Highlighting}[]
\NormalTok{circumference.in.cubits <-}\StringTok{ }\NormalTok{approx.pi}\OperatorTok{*}\NormalTok{diameter.in.cubits}
\NormalTok{circumference.in.cubits}
\end{Highlighting}
\end{Shaded}

\begin{verbatim}
## [1] 31.42857
\end{verbatim}

\begin{Shaded}
\begin{Highlighting}[]
\NormalTok{circumference.in.cubits <-}\StringTok{ }\DecValTok{30}
\NormalTok{circumference.in.cubits}
\end{Highlighting}
\end{Shaded}

\begin{verbatim}
## [1] 30
\end{verbatim}

\end{frame}

\section{Workspace}\label{workspace}

\begin{frame}{Using Names}

Using names and variables makes code: easier to design, easier to debug,
less prone to bugs, easier to improve, and easier for others to read

Avoid ``magic constants''; use named variables you will be graded on
this!

Named variables are a first step towards \textbf{abstraction}

\end{frame}

\begin{frame}[fragile]{The workspace}

What names have you defined values for?

\begin{Shaded}
\begin{Highlighting}[]
\KeywordTok{ls}\NormalTok{()}
\end{Highlighting}
\end{Shaded}

\begin{verbatim}
## [1] "approx.pi"               "circumference.in.cubits"
## [3] "diameter.in.cubits"
\end{verbatim}

\begin{Shaded}
\begin{Highlighting}[]
\KeywordTok{objects}\NormalTok{()}
\end{Highlighting}
\end{Shaded}

\begin{verbatim}
## [1] "approx.pi"               "circumference.in.cubits"
## [3] "diameter.in.cubits"
\end{verbatim}

Getting of variables:

\begin{Shaded}
\begin{Highlighting}[]
\KeywordTok{rm}\NormalTok{(}\StringTok{"circumference.in.cubits"}\NormalTok{)}
\KeywordTok{ls}\NormalTok{()}
\end{Highlighting}
\end{Shaded}

\begin{verbatim}
## [1] "approx.pi"          "diameter.in.cubits"
\end{verbatim}

\end{frame}

\begin{frame}[fragile]{First data structure: vectors}

Group related data values into one object, a \textbf{data structure}

A \textbf{vector} is a sequence of values, all of the same type

\begin{Shaded}
\begin{Highlighting}[]
\NormalTok{x <-}\StringTok{ }\KeywordTok{c}\NormalTok{(}\DecValTok{7}\NormalTok{, }\DecValTok{8}\NormalTok{, }\DecValTok{10}\NormalTok{, }\DecValTok{45}\NormalTok{)}
\NormalTok{x}
\end{Highlighting}
\end{Shaded}

\begin{verbatim}
## [1]  7  8 10 45
\end{verbatim}

\begin{Shaded}
\begin{Highlighting}[]
\KeywordTok{is.vector}\NormalTok{(x)}
\end{Highlighting}
\end{Shaded}

\begin{verbatim}
## [1] TRUE
\end{verbatim}

\texttt{c()} function returns a vector containing all its arguments in
order \texttt{x{[}1{]}} is the first element, \texttt{x{[}4{]}} is the
4th element\\
\texttt{x{[}-4{]}} is a vector containing all but the fourth element

\end{frame}

\begin{frame}[fragile]{}

\texttt{vector(length=6)} returns an empty vector of length 6; helpful
for filling things up later

\begin{Shaded}
\begin{Highlighting}[]
\NormalTok{weekly.hours <-}\StringTok{ }\KeywordTok{vector}\NormalTok{(}\DataTypeTok{length=}\DecValTok{5}\NormalTok{)}
\NormalTok{weekly.hours[}\DecValTok{5}\NormalTok{] <-}\StringTok{ }\DecValTok{8}
\end{Highlighting}
\end{Shaded}

\end{frame}

\begin{frame}[fragile]{Vector arithmetic}

Operators apply to vectors ``pairwise'' or ``elementwise'':

\begin{Shaded}
\begin{Highlighting}[]
\NormalTok{y <-}\StringTok{ }\KeywordTok{c}\NormalTok{(}\OperatorTok{-}\DecValTok{7}\NormalTok{, }\OperatorTok{-}\DecValTok{8}\NormalTok{, }\OperatorTok{-}\DecValTok{10}\NormalTok{, }\OperatorTok{-}\DecValTok{45}\NormalTok{)}
\NormalTok{x}\OperatorTok{+}\NormalTok{y}
\end{Highlighting}
\end{Shaded}

\begin{verbatim}
## [1] 0 0 0 0
\end{verbatim}

\begin{Shaded}
\begin{Highlighting}[]
\NormalTok{x}\OperatorTok{*}\NormalTok{y}
\end{Highlighting}
\end{Shaded}

\begin{verbatim}
## [1]   -49   -64  -100 -2025
\end{verbatim}

\end{frame}

\begin{frame}[fragile]{Recycling}

\textbf{Recycling} repeat elements in shorter vector when combined with
longer

\begin{Shaded}
\begin{Highlighting}[]
\NormalTok{x }\OperatorTok{+}\StringTok{ }\KeywordTok{c}\NormalTok{(}\OperatorTok{-}\DecValTok{7}\NormalTok{,}\OperatorTok{-}\DecValTok{8}\NormalTok{)}
\end{Highlighting}
\end{Shaded}

\begin{verbatim}
## [1]  0  0  3 37
\end{verbatim}

\begin{Shaded}
\begin{Highlighting}[]
\NormalTok{x}\OperatorTok{^}\KeywordTok{c}\NormalTok{(}\DecValTok{1}\NormalTok{,}\DecValTok{0}\NormalTok{,}\OperatorTok{-}\DecValTok{1}\NormalTok{,}\FloatTok{0.5}\NormalTok{)}
\end{Highlighting}
\end{Shaded}

\begin{verbatim}
## [1] 7.000000 1.000000 0.100000 6.708204
\end{verbatim}

Single numbers are vectors of length 1 for purposes of recycling:

\begin{Shaded}
\begin{Highlighting}[]
\DecValTok{2}\OperatorTok{*}\NormalTok{x}
\end{Highlighting}
\end{Shaded}

\begin{verbatim}
## [1] 14 16 20 90
\end{verbatim}

\end{frame}

\begin{frame}[fragile]{}

Can also do pairwise comparisons:

\begin{Shaded}
\begin{Highlighting}[]
\NormalTok{x }\OperatorTok{>}\StringTok{ }\DecValTok{9}
\end{Highlighting}
\end{Shaded}

\begin{verbatim}
## [1] FALSE FALSE  TRUE  TRUE
\end{verbatim}

Note: returns Boolean vector

Boolean operators work elementwise:

\begin{Shaded}
\begin{Highlighting}[]
\NormalTok{(x }\OperatorTok{>}\StringTok{ }\DecValTok{9}\NormalTok{) }\OperatorTok{&}\StringTok{ }\NormalTok{(x }\OperatorTok{<}\StringTok{ }\DecValTok{20}\NormalTok{)}
\end{Highlighting}
\end{Shaded}

\begin{verbatim}
## [1] FALSE FALSE  TRUE FALSE
\end{verbatim}

\end{frame}

\begin{frame}[fragile]{}

To compare whole vectors, best to use \texttt{identical()} or
\texttt{all.equal()}:

\begin{Shaded}
\begin{Highlighting}[]
\NormalTok{x }\OperatorTok{==}\StringTok{ }\OperatorTok{-}\NormalTok{y}
\end{Highlighting}
\end{Shaded}

\begin{verbatim}
## [1] TRUE TRUE TRUE TRUE
\end{verbatim}

\begin{Shaded}
\begin{Highlighting}[]
\KeywordTok{identical}\NormalTok{(x,}\OperatorTok{-}\NormalTok{y)}
\end{Highlighting}
\end{Shaded}

\begin{verbatim}
## [1] TRUE
\end{verbatim}

\begin{Shaded}
\begin{Highlighting}[]
\KeywordTok{identical}\NormalTok{(}\KeywordTok{c}\NormalTok{(}\FloatTok{0.5}\OperatorTok{-}\FloatTok{0.3}\NormalTok{,}\FloatTok{0.3}\OperatorTok{-}\FloatTok{0.1}\NormalTok{),}\KeywordTok{c}\NormalTok{(}\FloatTok{0.3}\OperatorTok{-}\FloatTok{0.1}\NormalTok{,}\FloatTok{0.5}\OperatorTok{-}\FloatTok{0.3}\NormalTok{))}
\end{Highlighting}
\end{Shaded}

\begin{verbatim}
## [1] FALSE
\end{verbatim}

\begin{Shaded}
\begin{Highlighting}[]
\KeywordTok{all.equal}\NormalTok{(}\KeywordTok{c}\NormalTok{(}\FloatTok{0.5}\OperatorTok{-}\FloatTok{0.3}\NormalTok{,}\FloatTok{0.3}\OperatorTok{-}\FloatTok{0.1}\NormalTok{),}\KeywordTok{c}\NormalTok{(}\FloatTok{0.3}\OperatorTok{-}\FloatTok{0.1}\NormalTok{,}\FloatTok{0.5}\OperatorTok{-}\FloatTok{0.3}\NormalTok{))}
\end{Highlighting}
\end{Shaded}

\begin{verbatim}
## [1] TRUE
\end{verbatim}

\end{frame}

\section{Simple Functions}\label{simple-functions}

\begin{frame}[fragile]{Functions on vectors}

Lots of functions take vectors as arguments: - \texttt{mean()},
\texttt{median()}, \texttt{sd()}, \texttt{var()}, \texttt{max()},
\texttt{min()}, \texttt{length()}, \texttt{sum()}: return single numbers
- \texttt{sort()} returns a new vector - \texttt{hist()} takes a vector
of numbers and produces a histogram, a highly structured object, with
the side-effect of making a plot - Similarly \texttt{ecdf()} produces a
cumulative-density-function object - \texttt{summary()} gives a
five-number summary of numerical vectors - \texttt{any()} and
\texttt{all()} are useful on Boolean vectors

\end{frame}

\begin{frame}[fragile]{Addressing vectors}

Vector of indices:

\begin{Shaded}
\begin{Highlighting}[]
\NormalTok{x[}\KeywordTok{c}\NormalTok{(}\DecValTok{2}\NormalTok{,}\DecValTok{4}\NormalTok{)]}
\end{Highlighting}
\end{Shaded}

\begin{verbatim}
## [1]  8 45
\end{verbatim}

Vector of negative indices

\begin{Shaded}
\begin{Highlighting}[]
\NormalTok{x[}\KeywordTok{c}\NormalTok{(}\OperatorTok{-}\DecValTok{1}\NormalTok{,}\OperatorTok{-}\DecValTok{3}\NormalTok{)]}
\end{Highlighting}
\end{Shaded}

\begin{verbatim}
## [1]  8 45
\end{verbatim}

(why that, and not \texttt{8\ 10}?)

\end{frame}

\begin{frame}[fragile]{Boolean vector}

\begin{Shaded}
\begin{Highlighting}[]
\NormalTok{x[x}\OperatorTok{>}\DecValTok{9}\NormalTok{]}
\end{Highlighting}
\end{Shaded}

\begin{verbatim}
## [1] 10 45
\end{verbatim}

\begin{Shaded}
\begin{Highlighting}[]
\NormalTok{y[x}\OperatorTok{>}\DecValTok{9}\NormalTok{]}
\end{Highlighting}
\end{Shaded}

\begin{verbatim}
## [1] -10 -45
\end{verbatim}

\texttt{which()} turns a Boolean vector in vector of TRUE indices:

\begin{Shaded}
\begin{Highlighting}[]
\NormalTok{places <-}\StringTok{ }\KeywordTok{which}\NormalTok{(x }\OperatorTok{>}\StringTok{ }\DecValTok{9}\NormalTok{)}
\NormalTok{places}
\end{Highlighting}
\end{Shaded}

\begin{verbatim}
## [1] 3 4
\end{verbatim}

\begin{Shaded}
\begin{Highlighting}[]
\NormalTok{y[places]}
\end{Highlighting}
\end{Shaded}

\begin{verbatim}
## [1] -10 -45
\end{verbatim}

\end{frame}

\begin{frame}[fragile]{Named components}

You can give names to elements or components of vectors

\begin{Shaded}
\begin{Highlighting}[]
\KeywordTok{names}\NormalTok{(x) <-}\StringTok{ }\KeywordTok{c}\NormalTok{(}\StringTok{"v1"}\NormalTok{,}\StringTok{"v2"}\NormalTok{,}\StringTok{"v3"}\NormalTok{,}\StringTok{"fred"}\NormalTok{)}
\KeywordTok{names}\NormalTok{(x)}
\end{Highlighting}
\end{Shaded}

\begin{verbatim}
## [1] "v1"   "v2"   "v3"   "fred"
\end{verbatim}

\begin{Shaded}
\begin{Highlighting}[]
\NormalTok{x[}\KeywordTok{c}\NormalTok{(}\StringTok{"fred"}\NormalTok{,}\StringTok{"v1"}\NormalTok{)]}
\end{Highlighting}
\end{Shaded}

\begin{verbatim}
## fred   v1 
##   45    7
\end{verbatim}

note the labels in what R prints; not actually part of the value

\end{frame}

\begin{frame}[fragile]{}

\texttt{names(x)} is just another vector (of characters):

\begin{Shaded}
\begin{Highlighting}[]
\KeywordTok{names}\NormalTok{(y) <-}\StringTok{ }\KeywordTok{names}\NormalTok{(x)}
\KeywordTok{sort}\NormalTok{(}\KeywordTok{names}\NormalTok{(x))}
\end{Highlighting}
\end{Shaded}

\begin{verbatim}
## [1] "fred" "v1"   "v2"   "v3"
\end{verbatim}

\begin{Shaded}
\begin{Highlighting}[]
\KeywordTok{which}\NormalTok{(}\KeywordTok{names}\NormalTok{(x)}\OperatorTok{==}\StringTok{"fred"}\NormalTok{)}
\end{Highlighting}
\end{Shaded}

\begin{verbatim}
## [1] 4
\end{verbatim}

\end{frame}

\begin{frame}{Take-Aways}

\begin{itemize}
\tightlist
\item
  We write programs by composing functions to manipulate data
\item
  The basic data types let us represent Booleans, numbers, and
  characters
\item
  Data structure let us group related values together
\item
  Vectors let us group values of the same type
\item
  Use variables rather a profusion of magic constants
\item
  Name components of structures to make data more meaningful
\end{itemize}

\end{frame}

\begin{frame}[fragile]{Peculiarites of floating-point numbers}

The more bits in the fraction part, the more precision

The R floating-point data type is a \texttt{double}, a.k.a.
\texttt{numeric} back when memory was expensive, the now-standard number
of bits was twice the default

Finite precision \(\Rightarrow\) arithmetic on \texttt{doubles} \(\neq\)
arithmetic on \(\mathbb{R}\).

\end{frame}

\begin{frame}[fragile]{}

\begin{Shaded}
\begin{Highlighting}[]
\FloatTok{0.45} \OperatorTok{==}\StringTok{ }\DecValTok{3}\OperatorTok{*}\FloatTok{0.15}
\end{Highlighting}
\end{Shaded}

\begin{verbatim}
## [1] FALSE
\end{verbatim}

\begin{Shaded}
\begin{Highlighting}[]
\FloatTok{0.45} \OperatorTok{-}\StringTok{ }\DecValTok{3}\OperatorTok{*}\FloatTok{0.15}
\end{Highlighting}
\end{Shaded}

\begin{verbatim}
## [1] 5.551115e-17
\end{verbatim}

\end{frame}

\begin{frame}[fragile]{}

Often ignorable, but not always - Rounding errors tend to accumulate in
long calculations - When results should be \(\approx 0\), errors can
flip signs - Usually better to use \texttt{all.equal()} than exact
comparison

\begin{Shaded}
\begin{Highlighting}[]
\NormalTok{(}\FloatTok{0.5} \OperatorTok{-}\StringTok{ }\FloatTok{0.3}\NormalTok{) }\OperatorTok{==}\StringTok{ }\NormalTok{(}\FloatTok{0.3} \OperatorTok{-}\StringTok{ }\FloatTok{0.1}\NormalTok{)}
\end{Highlighting}
\end{Shaded}

\begin{verbatim}
## [1] FALSE
\end{verbatim}

\begin{Shaded}
\begin{Highlighting}[]
\KeywordTok{all.equal}\NormalTok{(}\FloatTok{0.5}\OperatorTok{-}\FloatTok{0.3}\NormalTok{, }\FloatTok{0.3}\OperatorTok{-}\FloatTok{0.1}\NormalTok{)}
\end{Highlighting}
\end{Shaded}

\begin{verbatim}
## [1] TRUE
\end{verbatim}

\end{frame}

\begin{frame}[fragile]{Peculiarities of Integers}

Typing a whole number in the terminal doesn't make an integer; it makes
a double, whose fractional part is 0

\begin{Shaded}
\begin{Highlighting}[]
\KeywordTok{is.integer}\NormalTok{(}\DecValTok{7}\NormalTok{)}
\end{Highlighting}
\end{Shaded}

\begin{verbatim}
## [1] FALSE
\end{verbatim}

This looks like an integer

\begin{Shaded}
\begin{Highlighting}[]
\KeywordTok{as.integer}\NormalTok{(}\DecValTok{7}\NormalTok{)}
\end{Highlighting}
\end{Shaded}

\begin{verbatim}
## [1] 7
\end{verbatim}

To test for being a whole number, use \texttt{round()}:

\begin{Shaded}
\begin{Highlighting}[]
\KeywordTok{round}\NormalTok{(}\DecValTok{7}\NormalTok{) }\OperatorTok{==}\StringTok{ }\DecValTok{7}
\end{Highlighting}
\end{Shaded}

\begin{verbatim}
## [1] TRUE
\end{verbatim}

\end{frame}

\section{Thank You}\label{thank-you}

\end{document}
