\documentclass[8pt,ignorenonframetext,]{beamer}
\setbeamertemplate{caption}[numbered]
\setbeamertemplate{caption label separator}{: }
\setbeamercolor{caption name}{fg=normal text.fg}
\beamertemplatenavigationsymbolsempty
\usepackage{lmodern}
\usepackage{amssymb,amsmath}
\usepackage{ifxetex,ifluatex}
\usepackage{fixltx2e} % provides \textsubscript
\ifnum 0\ifxetex 1\fi\ifluatex 1\fi=0 % if pdftex
  \usepackage[T1]{fontenc}
  \usepackage[utf8]{inputenc}
\else % if luatex or xelatex
  \ifxetex
    \usepackage{mathspec}
  \else
    \usepackage{fontspec}
  \fi
  \defaultfontfeatures{Ligatures=TeX,Scale=MatchLowercase}
\fi
\usetheme[]{SDATbeamer}
% use upquote if available, for straight quotes in verbatim environments
\IfFileExists{upquote.sty}{\usepackage{upquote}}{}
% use microtype if available
\IfFileExists{microtype.sty}{%
\usepackage{microtype}
\UseMicrotypeSet[protrusion]{basicmath} % disable protrusion for tt fonts
}{}
\newif\ifbibliography
\hypersetup{
            pdftitle={Numerical Optimization},
            pdfauthor={S. Morteza Najibi, Shiraz University},
            pdfborder={0 0 0},
            breaklinks=true}
\urlstyle{same}  % don't use monospace font for urls
\usepackage{color}
\usepackage{fancyvrb}
\newcommand{\VerbBar}{|}
\newcommand{\VERB}{\Verb[commandchars=\\\{\}]}
\DefineVerbatimEnvironment{Highlighting}{Verbatim}{commandchars=\\\{\}}
% Add ',fontsize=\small' for more characters per line
\usepackage{framed}
\definecolor{shadecolor}{RGB}{248,248,248}
\newenvironment{Shaded}{\begin{snugshade}}{\end{snugshade}}
\newcommand{\KeywordTok}[1]{\textcolor[rgb]{0.13,0.29,0.53}{\textbf{#1}}}
\newcommand{\DataTypeTok}[1]{\textcolor[rgb]{0.13,0.29,0.53}{#1}}
\newcommand{\DecValTok}[1]{\textcolor[rgb]{0.00,0.00,0.81}{#1}}
\newcommand{\BaseNTok}[1]{\textcolor[rgb]{0.00,0.00,0.81}{#1}}
\newcommand{\FloatTok}[1]{\textcolor[rgb]{0.00,0.00,0.81}{#1}}
\newcommand{\ConstantTok}[1]{\textcolor[rgb]{0.00,0.00,0.00}{#1}}
\newcommand{\CharTok}[1]{\textcolor[rgb]{0.31,0.60,0.02}{#1}}
\newcommand{\SpecialCharTok}[1]{\textcolor[rgb]{0.00,0.00,0.00}{#1}}
\newcommand{\StringTok}[1]{\textcolor[rgb]{0.31,0.60,0.02}{#1}}
\newcommand{\VerbatimStringTok}[1]{\textcolor[rgb]{0.31,0.60,0.02}{#1}}
\newcommand{\SpecialStringTok}[1]{\textcolor[rgb]{0.31,0.60,0.02}{#1}}
\newcommand{\ImportTok}[1]{#1}
\newcommand{\CommentTok}[1]{\textcolor[rgb]{0.56,0.35,0.01}{\textit{#1}}}
\newcommand{\DocumentationTok}[1]{\textcolor[rgb]{0.56,0.35,0.01}{\textbf{\textit{#1}}}}
\newcommand{\AnnotationTok}[1]{\textcolor[rgb]{0.56,0.35,0.01}{\textbf{\textit{#1}}}}
\newcommand{\CommentVarTok}[1]{\textcolor[rgb]{0.56,0.35,0.01}{\textbf{\textit{#1}}}}
\newcommand{\OtherTok}[1]{\textcolor[rgb]{0.56,0.35,0.01}{#1}}
\newcommand{\FunctionTok}[1]{\textcolor[rgb]{0.00,0.00,0.00}{#1}}
\newcommand{\VariableTok}[1]{\textcolor[rgb]{0.00,0.00,0.00}{#1}}
\newcommand{\ControlFlowTok}[1]{\textcolor[rgb]{0.13,0.29,0.53}{\textbf{#1}}}
\newcommand{\OperatorTok}[1]{\textcolor[rgb]{0.81,0.36,0.00}{\textbf{#1}}}
\newcommand{\BuiltInTok}[1]{#1}
\newcommand{\ExtensionTok}[1]{#1}
\newcommand{\PreprocessorTok}[1]{\textcolor[rgb]{0.56,0.35,0.01}{\textit{#1}}}
\newcommand{\AttributeTok}[1]{\textcolor[rgb]{0.77,0.63,0.00}{#1}}
\newcommand{\RegionMarkerTok}[1]{#1}
\newcommand{\InformationTok}[1]{\textcolor[rgb]{0.56,0.35,0.01}{\textbf{\textit{#1}}}}
\newcommand{\WarningTok}[1]{\textcolor[rgb]{0.56,0.35,0.01}{\textbf{\textit{#1}}}}
\newcommand{\AlertTok}[1]{\textcolor[rgb]{0.94,0.16,0.16}{#1}}
\newcommand{\ErrorTok}[1]{\textcolor[rgb]{0.64,0.00,0.00}{\textbf{#1}}}
\newcommand{\NormalTok}[1]{#1}
\usepackage{graphicx,grffile}
\makeatletter
\def\maxwidth{\ifdim\Gin@nat@width>\linewidth\linewidth\else\Gin@nat@width\fi}
\def\maxheight{\ifdim\Gin@nat@height>\textheight0.8\textheight\else\Gin@nat@height\fi}
\makeatother
% Scale images if necessary, so that they will not overflow the page
% margins by default, and it is still possible to overwrite the defaults
% using explicit options in \includegraphics[width, height, ...]{}
\setkeys{Gin}{width=\maxwidth,height=\maxheight,keepaspectratio}

% Prevent slide breaks in the middle of a paragraph:
\widowpenalties 1 10000
\raggedbottom

\AtBeginPart{
  \let\insertpartnumber\relax
  \let\partname\relax
  \frame{\partpage}
}
\AtBeginSection{
  \ifbibliography
  \else
    \let\insertsectionnumber\relax
    \let\sectionname\relax
    \frame{\sectionpage}
  \fi
}
\AtBeginSubsection{
  \let\insertsubsectionnumber\relax
  \let\subsectionname\relax
  \frame{\subsectionpage}
}

\setlength{\parindent}{0pt}
\setlength{\parskip}{6pt plus 2pt minus 1pt}
\setlength{\emergencystretch}{3em}  % prevent overfull lines
\providecommand{\tightlist}{%
  \setlength{\itemsep}{0pt}\setlength{\parskip}{0pt}}
\setcounter{secnumdepth}{0}

\title{Numerical Optimization}
\author{S. Morteza Najibi, Shiraz University}
\date{May 25, 2019}

\begin{document}
\frame{\titlepage}

\begin{frame}[fragile]

\#\# Agenda

\begin{itemize}
\tightlist
\item
  Basics of optimization
\item
  Gradient descent
\item
  Newton's method
\item
  Curve-fitting
\item
  R: \texttt{optim}, \texttt{nls}
\end{itemize}

\end{frame}

\begin{frame}{Examples of Optimization Problems}

\begin{itemize}
\tightlist
\item
  Minimize mean-squared error of regression surface
\item
  Maximize likelihood of distribution
\item
  Maximize output of tanks from given supplies and factories;
\item
  Maximize return of portfolio for given volatility
\item
  Minimize cost of airline flight schedule
\end{itemize}

\end{frame}

\begin{frame}{Optimization Problems}

Given an \textbf{objective function} \(f: \mathcal{D} \mapsto R\), find

\[
\DeclareMathOperator*{\argmax}{argmax}
\DeclareMathOperator*{\argmin}{argmin}
\theta^* = \argmin_{\theta}{f(\theta)}
\]

Basics: maximizing \(f\) is minimizing \(-f\):

\[
\argmax_{\theta}{f(\theta)}= \argmin_{\theta}{-f(\theta)}
\]

If \(h\) is strictly increasing (e.g., \(\log\)), then

\[
\argmin_{\theta}{f(\theta)} = \argmin_{\theta}{h(f(\theta))}
\]

\end{frame}

\begin{frame}{Considerations}

\begin{itemize}
\item
  Approximation: How close can we get to \(\theta^*\), and/or
  \(f(\theta^*)\)?
\item
  Time complexity: How many computer steps does that take?\\
  Varies with precision of approximation, niceness of \(f\), size of
  \(\mathcal{D}\), size of data, method\ldots{}
\item
  Most optimization algorithms use \textbf{successive approximation}, so
  distinguish number of iterations from cost of each iteration
\end{itemize}

\end{frame}

\begin{frame}{You remember calculus, right?}

Suppose \(x\) is one dimensional and \(f\) is smooth. If \(x^*\) is an
\textbf{interior} minimum / maximum / extremum point

\[
{\left. \frac{df}{dx} \right|}_{x=x^*} = 0
\]

If \(x^*\) a minimum, \[
{\left. \frac{d^2f}{dx^2}\right|}_{x=x^*} > 0
\]

\end{frame}

\begin{frame}

This all carries over to multiple dimensions:

At an \textbf{interior extremum}, \[
\nabla f(\theta^*) = 0
\]

At an \textbf{interior minimum}, \[
\nabla^2 f(\theta^*) \geq 0
\] meaning for any vector \(v\), \[
v^T \nabla^2 f(\theta^*) v \geq 0
\] \(\nabla^2 f =\) the \textbf{Hessian}, \(\mathbf{H}\)

\(\theta\) might just be a \textbf{local} minimum

\end{frame}

\begin{frame}{Gradients and Changes to f}

\[
f^{\prime}(x_0)  =  {\left. \frac{df}{dx}\right|}_{x=x_0} = \lim_{x\rightarrow x_0}{\frac{f(x)-f(x_0)}{x-x_0}} \]

\[    f(x)  \approx f(x_0) +(x-x_0)f^{\prime}(x_0)
\]

Locally, the function looks linear; to minimize a linear function, move
down the slope

Multivariate version: \[
f(\theta) \approx f(\theta_0) + (\theta-\theta_0) \cdot \nabla f(\theta_0)
\]

\(\nabla f(\theta_0)\) points in the direction of fastest ascent at
\(\theta_0\)

\end{frame}

\begin{frame}{Gradient Descent}

\begin{enumerate}
\def\labelenumi{\arabic{enumi}.}
\tightlist
\item
  Start with initial guess for \(\theta\), step-size \(\eta\)
\item
  While ((not too tired) and (making adequate progress))
\end{enumerate}

\begin{itemize}
\tightlist
\item
  Find gradient \(\nabla f(\theta)\)
\item
  Set \(\theta \leftarrow \theta - \eta \nabla f(\theta)\)
\end{itemize}

\begin{enumerate}
\def\labelenumi{\arabic{enumi}.}
\setcounter{enumi}{2}
\tightlist
\item
  Return final \(\theta\) as approximate \(\theta^*\)
\end{enumerate}

Variations: adaptively adjust \(\eta\) to make sure of improvement or
search along the gradient direction for minimum

\end{frame}

\begin{frame}{Taylor Series}

What if we do a quadratic approximation to \(f\)?

\[
f(x) \approx f(x_0) + (x-x_0)f^{\prime}(x_0) + \frac{1}{2}(x-x_0)^2
f^{\prime\prime}(x_0)
\]

Special cases of general idea of Taylor approximation

Simplifies if \(x_0\) is a minimum since then \(f^{\prime}(x_0) = 0\):

\[
f(x) \approx f(x_0) + \frac{1}{2}(x-x_0)^2 f^{\prime\prime}(x_0)
\]

Near a minimum, smooth functions look like parabolas

Carries over to the multivariate case: \[
f(\theta) \approx f(\theta_0) + (\theta-\theta_0) \cdot \nabla f(\theta_0) +
\frac{1}{2}(\theta-\theta_0)^T \mathbf{H}(\theta_0) (\theta-\theta_0)
\]

\end{frame}

\begin{frame}{Minimizing a Quadratic}

If we know \[
f(x) = ax^2 + bx + c
\]

we minimize exactly:

\[
\begin{eqnarray*}
2ax^* + b & = & 0\\
x^* & = & \frac{-b}{2a}
\end{eqnarray*}
\]

If \[
f(x) = \frac{1}{2}a (x-x_0)^2 + b(x-x_0) + c
\]

then \[
x^* = x_0 - a^{-1}b
\]

\end{frame}

\begin{frame}{Newton's Method}

Taylor-expand for the value \emph{at the minimum} \(\theta^*\) \[
    f(\theta^*) \approx f(\theta) + (\theta^*-\theta) \nabla f(\theta) +
      \frac{1}{2}(\theta^*-\theta)^T \mathbf{H}(\theta) (\theta^*-\theta)
    \]

Take gradient, set to zero, solve for \(\theta^*\): \[
    \begin{eqnarray*}
    0 & = & \nabla f(\theta) + \mathbf{H}(\theta) (\theta^*-\theta) \\
    \theta^* & = & \theta - {\left(\mathbf{H}(\theta)\right)}^{-1} \nabla f(\theta)
    \end{eqnarray*}
    \]

Works \emph{exactly} if \(f\) is quadratic\\
and \(\mathbf{H}^{-1}\) exists, etc.

If \(f\) isn't quadratic, keep pretending it is until we get close to
\(\theta^*\), when it will be nearly true

\end{frame}

\begin{frame}{Newton's Method: The Algorithm}

\begin{enumerate}
\def\labelenumi{\arabic{enumi}.}
\tightlist
\item
  Start with guess for \(\theta\)
\item
  While ((not too tired) and (making adequate progress))
\end{enumerate}

\begin{itemize}
\tightlist
\item
  Find gradient \(\nabla f(\theta)\) and Hessian \(\mathbf{H}(\theta)\)
\item
  Set
  \(\theta \leftarrow \theta - \mathbf{H}(\theta)^{-1} \nabla f(\theta)\)
\end{itemize}

\begin{enumerate}
\def\labelenumi{\arabic{enumi}.}
\setcounter{enumi}{2}
\tightlist
\item
  Return final \(\theta\) as approximation to \(\theta^*\)
\end{enumerate}

Like gradient descent, but with inverse Hessian giving the step-size

``This is about how far you can go with that gradient''

\#\# Curve-Fitting by Optimizing

We have data \((x_1, y_1), (x_2, y_2), \ldots (x_n, y_n)\)

We also have possible curves, \(r(x;\theta)\)

e.g., \(r(x) = x \cdot \theta\)

e.g., \(r(x) = \theta_1 x^{\theta_2}\)

\#\# Curve-Fitting by Optimizing

Least-squares curve fitting:

\[
    \hat{\theta} = \argmin_{\theta}{\frac{1}{n}\sum_{i=1}^n{(y_i - r(x_i;\theta))^2}}
    \]

``Robust'' curve fitting: \[
    \hat{\theta} = \argmin_{\theta}{\frac{1}{n}\sum_{i=1}^{n}{\psi(y_i - r(x_i;\theta))}}
    \]

\end{frame}

\begin{frame}[fragile]{Derivatives in R}

\begin{Shaded}
\begin{Highlighting}[]
\NormalTok{f <-}\StringTok{ }\KeywordTok{expression}\NormalTok{(}\DecValTok{5}\OperatorTok{*}\NormalTok{x}\OperatorTok{^}\DecValTok{2}\OperatorTok{+}\DecValTok{3}\OperatorTok{*}\NormalTok{x}\OperatorTok{+}\DecValTok{1}\NormalTok{) ; }\KeywordTok{deriv}\NormalTok{(f,}\StringTok{"x"}\NormalTok{)}
\end{Highlighting}
\end{Shaded}

\begin{verbatim}
## expression({
##     .value <- 5 * x^2 + 3 * x + 1
##     .grad <- array(0, c(length(.value), 1L), list(NULL, c("x")))
##     .grad[, "x"] <- 5 * (2 * x) + 3
##     attr(.value, "gradient") <- .grad
##     .value
## })
\end{verbatim}

\begin{Shaded}
\begin{Highlighting}[]
\NormalTok{x <-}\StringTok{ }\DecValTok{2}\OperatorTok{:}\DecValTok{4}\NormalTok{ ; }\KeywordTok{eval}\NormalTok{(}\KeywordTok{deriv}\NormalTok{(f,}\StringTok{"x"}\NormalTok{))}
\end{Highlighting}
\end{Shaded}

\begin{verbatim}
## [1] 27 55 93
## attr(,"gradient")
##       x
## [1,] 23
## [2,] 33
## [3,] 43
\end{verbatim}

\end{frame}

\begin{frame}[fragile]{Integrate in R}

\begin{Shaded}
\begin{Highlighting}[]
\NormalTok{g <-}\StringTok{ }\ControlFlowTok{function}\NormalTok{(x) }\KeywordTok{eval}\NormalTok{(f)}
\KeywordTok{integrate}\NormalTok{(g,}\DataTypeTok{lower =} \DecValTok{0}\NormalTok{,}\DataTypeTok{upper =} \DecValTok{2}\NormalTok{)}
\end{Highlighting}
\end{Shaded}

\begin{verbatim}
## 21.33333 with absolute error < 2.4e-13
\end{verbatim}

\begin{Shaded}
\begin{Highlighting}[]
\NormalTok{h <-}\StringTok{ }\ControlFlowTok{function}\NormalTok{(x) }\KeywordTok{exp}\NormalTok{(}\KeywordTok{cos}\NormalTok{(x)}\OperatorTok{^}\DecValTok{2}\NormalTok{)}\OperatorTok{+}\KeywordTok{exp}\NormalTok{(}\KeywordTok{sin}\NormalTok{(x))}
\KeywordTok{integrate}\NormalTok{(h,}\DecValTok{1}\NormalTok{,}\DecValTok{5}\NormalTok{)}
\end{Highlighting}
\end{Shaded}

\begin{verbatim}
## 11.99573 with absolute error < 0.00046
\end{verbatim}

\end{frame}

\begin{frame}[fragile]{Optimization in R: optim()}

\begin{verbatim}
optim(par, fn, gr, method, control, hessian)
\end{verbatim}

\begin{itemize}
\tightlist
\item
  \texttt{fn}: function to be minimized; mandatory
\item
  \texttt{par}: initial parameter guess; mandatory
\item
  \texttt{gr}: gradient function; only needed for some methods
\item
  \texttt{method}: defaults to a gradient-free method (``Nedler-Mead''),
  could be BFGS (Newton-ish)
\item
  \texttt{control}: optional list of control settings
\item
  (maximum iterations, scaling, tolerance for convergence, etc.)
\item
  \texttt{hessian}: should the final Hessian be returned? default FALSE
\end{itemize}

Return contains the location (\texttt{\$par}) and the value
(\texttt{\$val}) of the optimum, diagnostics, possibly
\texttt{\$hessian}

\end{frame}

\begin{frame}[fragile]{Optimization in R: optim()}

\begin{Shaded}
\begin{Highlighting}[]
\NormalTok{gmp <-}\StringTok{ }\KeywordTok{read.table}\NormalTok{(}\StringTok{"../data/gmp.dat"}\NormalTok{)}
\NormalTok{gmp}\OperatorTok{$}\NormalTok{pop <-}\StringTok{ }\NormalTok{gmp}\OperatorTok{$}\NormalTok{gmp}\OperatorTok{/}\NormalTok{gmp}\OperatorTok{$}\NormalTok{pcgmp}
\KeywordTok{library}\NormalTok{(numDeriv)}
\NormalTok{mse <-}\StringTok{ }\ControlFlowTok{function}\NormalTok{(theta) \{ }\KeywordTok{mean}\NormalTok{((gmp}\OperatorTok{$}\NormalTok{pcgmp }\OperatorTok{-}\StringTok{ }\NormalTok{theta[}\DecValTok{1}\NormalTok{]}\OperatorTok{*}\NormalTok{gmp}\OperatorTok{$}\NormalTok{pop}\OperatorTok{^}\NormalTok{theta[}\DecValTok{2}\NormalTok{])}\OperatorTok{^}\DecValTok{2}\NormalTok{) \}}
\NormalTok{grad.mse <-}\StringTok{ }\ControlFlowTok{function}\NormalTok{(theta) \{ }\KeywordTok{grad}\NormalTok{(}\DataTypeTok{func=}\NormalTok{mse,}\DataTypeTok{x=}\NormalTok{theta) \}}
\NormalTok{theta0=}\KeywordTok{c}\NormalTok{(}\DecValTok{5000}\NormalTok{,}\FloatTok{0.15}\NormalTok{)}
\NormalTok{fit1 <-}\StringTok{ }\KeywordTok{optim}\NormalTok{(theta0,mse,grad.mse,}\DataTypeTok{method=}\StringTok{"BFGS"}\NormalTok{,}\DataTypeTok{hessian=}\OtherTok{TRUE}\NormalTok{)}
\end{Highlighting}
\end{Shaded}

\end{frame}

\begin{frame}[fragile]{fit1: Newton-ish BFGS method}

\begin{Shaded}
\begin{Highlighting}[]
\NormalTok{fit1[}\DecValTok{1}\OperatorTok{:}\DecValTok{3}\NormalTok{]}
\end{Highlighting}
\end{Shaded}

\begin{verbatim}
## $par
## [1] 6493.2563738    0.1276921
## 
## $value
## [1] 61853983
## 
## $counts
## function gradient 
##       63       11
\end{verbatim}

\end{frame}

\begin{frame}[fragile]{fit1: Newton-ish BFGS method}

\begin{Shaded}
\begin{Highlighting}[]
\NormalTok{fit1[}\DecValTok{4}\OperatorTok{:}\DecValTok{6}\NormalTok{]}
\end{Highlighting}
\end{Shaded}

\begin{verbatim}
## $convergence
## [1] 0
## 
## $message
## NULL
## 
## $hessian
##             [,1]         [,2]
## [1,] 5.25021e+01      4422070
## [2,] 4.42207e+06 375729087977
\end{verbatim}

\end{frame}

\begin{frame}[fragile]{nls}

\texttt{optim} is a general-purpose optimizer

So is \texttt{nlm} --- try them both if one doesn't work

\texttt{nls} is for nonlinear least squares

\end{frame}

\begin{frame}[fragile]{nls}

\begin{verbatim}
nls(formula, data, start, control, [[many other options]])
\end{verbatim}

\begin{itemize}
\tightlist
\item
  \texttt{formula}: Mathematical expression with response variable,
  predictor variable(s), and unknown parameter(s)
\item
  \texttt{data}: Data frame with variable names matching
  \texttt{formula}
\item
  \texttt{start}: Guess at parameters (optional)
\item
  \texttt{control}: Like with \texttt{optim} (optional)
\end{itemize}

Returns an \texttt{nls} object, with fitted values, prediction methods,
etc.

The default optimization is a version of Newton's method

\end{frame}

\begin{frame}[fragile]{fit2: Fitting the Same Model with nls()}

\begin{Shaded}
\begin{Highlighting}[]
\NormalTok{fit2 <-}\StringTok{ }\KeywordTok{nls}\NormalTok{(pcgmp}\OperatorTok{~}\NormalTok{y0}\OperatorTok{*}\NormalTok{pop}\OperatorTok{^}\NormalTok{a,}\DataTypeTok{data=}\NormalTok{gmp,}\DataTypeTok{start=}\KeywordTok{list}\NormalTok{(}\DataTypeTok{y0=}\DecValTok{5000}\NormalTok{,}\DataTypeTok{a=}\FloatTok{0.1}\NormalTok{))}
\KeywordTok{summary}\NormalTok{(fit2)}
\end{Highlighting}
\end{Shaded}

\begin{verbatim}
## 
## Formula: pcgmp ~ y0 * pop^a
## 
## Parameters:
##     Estimate Std. Error t value Pr(>|t|)    
## y0 6.494e+03  8.565e+02   7.582 2.87e-13 ***
## a  1.277e-01  1.012e-02  12.612  < 2e-16 ***
## ---
## Signif. codes:  0 '***' 0.001 '**' 0.01 '*' 0.05 '.' 0.1 ' ' 1
## 
## Residual standard error: 7886 on 364 degrees of freedom
## 
## Number of iterations to convergence: 5 
## Achieved convergence tolerance: 1.751e-07
\end{verbatim}

\end{frame}

\begin{frame}[fragile]{fit2: Fitting the Same Model with nls()}

\begin{Shaded}
\begin{Highlighting}[]
\KeywordTok{plot}\NormalTok{(pcgmp}\OperatorTok{~}\NormalTok{pop,}\DataTypeTok{data=}\NormalTok{gmp)}
\NormalTok{pop.order <-}\StringTok{ }\KeywordTok{order}\NormalTok{(gmp}\OperatorTok{$}\NormalTok{pop)}
\KeywordTok{lines}\NormalTok{(gmp}\OperatorTok{$}\NormalTok{pop[pop.order],}\KeywordTok{fitted}\NormalTok{(fit2)[pop.order])}
\KeywordTok{curve}\NormalTok{(fit1}\OperatorTok{$}\NormalTok{par[}\DecValTok{1}\NormalTok{]}\OperatorTok{*}\NormalTok{x}\OperatorTok{^}\NormalTok{fit1}\OperatorTok{$}\NormalTok{par[}\DecValTok{2}\NormalTok{],}\DataTypeTok{add=}\OtherTok{TRUE}\NormalTok{,}\DataTypeTok{lty=}\StringTok{"dashed"}\NormalTok{,}\DataTypeTok{col=}\StringTok{"blue"}\NormalTok{)}
\end{Highlighting}
\end{Shaded}

\includegraphics{Section10_files/figure-beamer/unnamed-chunk-7-1.pdf}

\end{frame}

\begin{frame}[fragile]{Maximize log-likelihood}

Imagine that we have a sample that was drawn from a normal distribution
with unknown mean, \(\mu\), and variance, \(\sigma^2\). The objective is
to estimate these parameters. The normal log-likelihood function is
given by

\[ l = −.5 n ln (2\pi)−.5 n ln(\sigma^2)− \frac{1}{2\sigma^2} \sum_{i} (y_i -\mu)^2 \]
We can program this function in the following way:

\texttt{r\ \ \ normal.lik\textless{}-function(theta,y)\ \{\ \ \ \ mu\textless{}-theta{[}1{]}\ \ \ sigma2\textless{}-theta{[}2{]}\ \ \ n\ \textless{}-\ length(y)\ \ \ logl\ \textless{}-\ -.5*n*log(2*pi)\ -.5*n*log(sigma2)\ -\ (1/(2*sigma2))*sum((y-mu)**2)\ \ \ \ return\ (-logl)\ \ \ \}}

\end{frame}

\begin{frame}[fragile]

Once the log-likelihood function has been declared, then the optim
command can be invoked. The minimal specification of this command is

\texttt{optim(starting\ values,\ log-likelihood,\ data)}

Given a vector of data, y, the parameters of the normal distrib- ution
can be estimated using

\begin{Shaded}
\begin{Highlighting}[]
\NormalTok{y <-}\StringTok{ }\KeywordTok{rnorm}\NormalTok{(}\DecValTok{1000}\NormalTok{,}\DecValTok{1}\NormalTok{,}\DecValTok{5}\NormalTok{);}
\KeywordTok{optim}\NormalTok{(}\KeywordTok{c}\NormalTok{(}\DecValTok{0}\NormalTok{,}\DecValTok{1}\NormalTok{),normal.lik,}\DataTypeTok{y=}\NormalTok{y,}\DataTypeTok{method=}\StringTok{"BFGS"}\NormalTok{)}
\end{Highlighting}
\end{Shaded}

\begin{verbatim}
## $par
## [1]  1.00050 24.42953
## 
## $value
## [1] 3016.806
## 
## $counts
## function gradient 
##       40       25 
## 
## $convergence
## [1] 0
## 
## $message
## NULL
\end{verbatim}

\end{frame}

\begin{frame}[fragile]{Summary}

\begin{enumerate}
\def\labelenumi{\arabic{enumi}.}
\tightlist
\item
  Trade-offs: complexity of iteration vs.~number of iterations
  vs.~precision of approximation
\end{enumerate}

\begin{itemize}
\tightlist
\item
  Gradient descent: less complex iterations, more guarantees, less
  adaptive
\item
  Newton: more complex iterations, but few of them for good functions,
  more adaptive, less robust
\end{itemize}

\begin{enumerate}
\def\labelenumi{\arabic{enumi}.}
\setcounter{enumi}{1}
\tightlist
\item
  Start with pre-built code like \texttt{optim} or \texttt{nls},
  implement your own as needed
\end{enumerate}

\end{frame}

\end{document}
